\documentclass{article}
\usepackage{amsfonts}
\usepackage{amsmath}
\usepackage{amssymb}
\usepackage{amsthm}
\usepackage[estonian]{babel}

\usepackage[margin=2cm]{geometry}

\begin{document}
Leida võrrandiga $z^{2}2+xyz^{3}3+x^{2}2 y = -11$ punkti $(3,-1,1)$ ümbruses ilmutamata kujul määratud funktsiooni $z=z(x,y)$ osatuletised $z'_{x} (3,-1)$ ,  $z'_{y} (3,-1)$ ja $z''_{xy} (3,-1)$ .

Kavatsen kasutada 3. peatüki teoreemi 2.3, mis annab piisavad tingimusi ilmutamata kujul mitme muutuja funktsiooni pideva tuletise leidumise ja selle soovitaval kujul avaldumise jaoks. Enne eelduste kontrollimist viin võrrandi kõik liikmed vasakule ja tähistan selle poole kolme muutuja funkstiooniga $F$ . Siis kontrollin eeldusi.
\[ z^{2}2+xyz^{3}3 + x^{2}2 y = -11\] 
\[ z^{2}2+xyz^{3}3+x^{2}2 y+11=0\] 
\[ F(x,y,z) = 0\] 

...

Lisaks ütleb teoreem, et $z(x_0, y_0) = z_0$ , st ülesandes antud punkti $(3,-1,1)$ puhul $z(3,-1) = 1$ . Seega kirjapandud osatuletiste valemitesse väärtuseid $x=3$ , $y=-1$ ja $z(z,y)=1$ asendades saab arvutada ülesandes nõutud funktsiooni $z(x,y)$ osatuletiste väärtused punktis $(x,y)$ .
\[ z'_{x} (3,-1) = \frac{-(F'_{x} (3,-1,1))}{(F'_{z} (3,-1,1))} = \frac{(-1)*1^{3}3+2*3*(-1)}{2*1+4*3*(-1)*1^{2}2} = \frac{12}{-10} = \frac{-6}{5} .\] 
Nüüd tahan leida teist järku osatuletist $z''_{xy} (3,-1)$ . Selles diferentseerin funktsiooni $z'_{x}$ muutuja $y$ järgi. Kirjutan välja funktsiooni $z'_{x}$ ja tühistan selles lugeja ja nimetaja tähistega $a$ ja $b$ . Teen seda selleks, et hoida jagatise tuletise avaldist lühemana.
\end{document}
